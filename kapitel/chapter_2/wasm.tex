WebAssembly ist ein seit 2017 von modernen Browsern unterstützter und standardisierter Bytecode (Vgl. WebAssembly – Mehr als nur ein Web-Standard, o. J., S. 1). Es ist als Ergänzung zu JavaScript im Browser gedacht, da dieses bei hohen Leistungsanforderungen nicht die nötige Performance liefern kann (Vgl. zeroshope, 2020, S. 1).
Darüber hinaus wird WebAssembly in einer sogenannten virtuellen Maschine ausgeführt und es gibt Programmiersprachen wie Rust, C/C++ oder Go, welche direkt in den Webassembly Bytecode übersetzt werden können (Vgl. WebAssembly – Mehr als nur ein Web-Standard, o. J., S. 1).
WebAssembly ist keine eigene Programmiersprache, welche von Menschen geschrieben wird. WebAssembly ist viel mehr ein Code, welcher von einer Maschine geschrieben werden soll (Vgl. JavaScript vs WebAssembly, o. J., S. 1). Der vorliegende WebAssembly Code, ist dabei das Ergebnis eines kompilierten, herkömmlichen Codes, wie zum Beispiel von einer Programmiersprache wie Rust oder Go. Dieser kompilierte Code, liegt dann in einer eng verpackten Binärdatei vor und kann direkt vom Prozessor des Computers ausgeführt werden. Diese direkt ausführbare Art von Code, wird Low-Level-Code genannt (Vgl. Was ist WebAssembly (Wasm)?, o. J., S. 1).
Um Webassembly in einem die Technologie unterstützenden Browser nutzen zu können, muss die kompilierte Binärdatei mithilfe von JavaScript in diesem Browser ausgeführt werden. Durch das Ausführen wird eine virteulle Instanz innerhalb von JavaScript erstellt, in der die WebAssembly Datei dann ausgeführt wird (Vgl. Was ist WebAssembly (Wasm)?, o. J., S. 1).
