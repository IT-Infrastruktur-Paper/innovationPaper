Um das herausgearbeitet Potenzial des Konzepts vollständig auszuschöpfen, können folgende Erweiterungsmöglichkeiten betrachtet werden.

\textbf{Schnittstellen zu weiteren relevanten Behörden und Institutionen:} \\
Über die bereits genannten Institutionen hinaus existieren weitere, mit welchen Gründer im Laufe des Gründungsprozesses gegebenenfalls in Kontakt treten müssen.
Hierzu zählen beispielsweise im Kontext von Angestellten die Agentur für Arbeit und die Krankenkassen.
Außerdem gibt es branchenspezifische Sonderlocken wie die Gaststättenerlaubnis und die Eintragung in die Handwerksrolle.

\textbf{Automatisierte Dokumentenerstellung:} \\
Ein zentrales Element der Reduktion des administrativen Aufwands kann die automatisierte Dokumentenerstellung darstellen.
Durch die Integration von Vorlagen und intelligenten Algorithmen können standardisierte Dokumente wie Gesellschaftsverträge und Anträge auf Basis der vom Gründer bereitgestellten Informationen automatisiert erstellt werden.
Das minimiert sowohl den Zeitaufwand für Gründer als auch das Risiko von Fehlern und Inkonsistenzen in den Dokumenten.

\textbf{Vermittlung oder Integration von Finanz- und Versicherungsdienstleistungen:} \\
Die Finanzierung und Absicherung von unternehmerischen Risiken stellt für viele Gründer eine zentrale Herausforderung dar.
Die Integration direkter Schnittstellen zu Banken oder ein Angebot der Finanzierung durch die hauseigene Bank ermöglicht eine beschleunigte und vereinfachte Kreditvergabe.
Durch die Übermittlung relevanter Daten aus dem Portal an die Banken können automatisierte Bonitätsprüfungen und Kreditentscheidungen ermöglicht werden. Dies verkürzt den Prozess der Kreditbeantragung erheblich und bietet Gründern schneller Planungssicherheit.
Dasselbe Prinzip kann auch auf Versicherungsprodukte angewendet werden.

\textbf{Berücksichtigung unterschiedlicher Rechtssysteme:} \\
Die langfristige Vision des Projekts ist die Integration in unterschiedliche Rechtssysteme, dies ermöglicht sowohl die Erschließung von neuen Märkten als auch die Möglichkeit für internationale Unternehmen die administrativen Aufwände zu bündeln und zentral zu orchestrieren.
Zwei zentrale negative Standortfaktoren Europas sind die sprachliche und rechtliche Heterogenität. Eine fehlende europäische Rechtsform für Unternehmen erschwert Startups schnelle Skalierung, wie sie in den USA und in China möglich sind.
Das Projekt könnte diese Hürden überwindbarer machen. Allerdings ist dies ein hochkomplexes Unterfangen und bedarf einer langfristigen Entwicklung. Voraussetzung sind eine umfassende Analyse der jeweiligen Rechtssysteme, eine flexible und skalierbare Architektur des Portals und die kontinuierliche Zusammenarbeit mit internationalen Rechtsexperten und Institutionen.


