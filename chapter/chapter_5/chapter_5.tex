\newpage

\section{Fazit} \label{fazit}

Im Rahmen dieser Arbeit hat sich die Problematik der Unterstützung und Markenanmeldung in Deutschland und den damit verbundenen Herausforderungen, insbesondere der Digitalisierung gezeigt.
in der prozessualen Analyse der Unternehmens- \& Markengründung und den damit verbundenen digitalen Schnittstellen wurde deutlich, dass der Gründungsprozess in Deutschland trotz bestehender digitaler Möglichkeiten nach wie vor komplex, zeitaufwendig und wenig transparent ist. 
Die Vielzahl an involvierten Behörden und die fehlende vollständige, konsequente Digitalisierung führen zu einem hohen bürokratischen Aufwand für Gründer.

Die im Rahmen der Methodik angewandte Design-Thinking-Prozess ermöglichte es, die Bedürfnisse der Zielgruppe – angehende Unternehmer – zu identifizieren.

Zentrale Bedürfnisse sind ein einfacher Überblick über alle notwendigen Schritte, eine digitale Abwicklung aus einer Hand, schnelle Bearbeitungszeiten, digitale Identifikation und Bezahlung sowie Unterstützung bei der Wahl der Rechtsform und der Markenanmeldung.
Der Status der einzelnen Schritte sollte dabei jederzeit einsehbar sein.

Der entwickelte Lösungsvorschlag in Form eines Webportals adressiert diese Bedürfnisse. Das Portal soll als zentrale Anlaufstelle dienen, die den gesamten Gründung- / Umfirmingsprozess strukturiert, digitalisiert und visuell aufbereitet.
Durch die Bündelung nahezu  aller relevanten Behördenkontakte und die Integration digitaler Funktionen wie Identitätsfeststellung, Bezahlung und Statusabfrage wird der Prozess vereinfacht und beschleunigt.

Die Wettbewerberanalyse zeigte, dass zwar bereits einige Anbieter existieren, die Teile des Gründungsprozesses digitalisieren, jedoch keine, welche den gesamten Prozess Ende zu Ende abdecken.
Diese Lücke ist nicht nur bei den angebotenen Schnittstellen zu den relevanten Behörden ersichtlich, sondern auch bei den bei den verfügbaren Rechtsformen und der der Markeneintragung.

Abschließend ergibt sich daraus, dass ein solches Portal das Potenzial hat, den Unternehmensgründungsprozess in Deutschland signifikant zu vereinfachen und zu modernisieren.
Es könnte die bürokratische Belastung für Gründer reduzieren, die Transparenz erhöhen und die Gründungsbereitschaft fördern.
Das Ergebnis der Wettbewerberanalyse unterschtreicht das Marktpotenzial eines solchen Angebots, da bereits erfolgreiche Anbieter existieren, aber gleichzeitig nicht alle relevanten Aspekte des Gründungsprozesses abdecken.
Die Umsetzung des Konzepts erfordert eine enge Zusammenarbeit mit den beteiligten Behörden und Institutionen und eine sorgfältige technische Implentation, insbesondere in Hinblick auf Schnittstellen und Datensicherheit.


\subsection{Erweiterungsmöglichkeiten} \label{erweiterungsmoeglichkeiten}
Um das herausgearbeitet Potenzial des Konzepts vollständig auszuschöpfen, können folgende Erweiterungsmöglichkeiten betrachtet werden.

\textbf{Schnittstellen zu weiteren relevanten Behörden und Institutionen:} \\
Über die bereits genannten Institutionen hinaus existieren weitere, mit welchen Gründer im Laufe des Gründungsprozesses gegebenenfalls in Kontakt treten müssen.
Hierzu zählen beispielsweise im Kontext von Angestellten die Agentur für Arbeit und die Krankenkassen.
Außerdem gibt es branchenspezifische Sonderlocken wie die Gaststättenerlaubnis und die Eintragung in die Handwerksrolle.

\textbf{Automatisierte Dokumentenerstellung:} \\
Ein zentrales Element der Reduktion des administrativen Aufwands kann die automatisierte Dokumentenerstellung darstellen.
Durch die Integration von Vorlagen und intelligenten Algorithmen können standardisierte Dokumente wie Gesellschaftsverträge und Anträge auf Basis der vom Gründer bereitgestellten Informationen automatisiert erstellt werden.
Das minimiert sowohl den Zeitaufwand für Gründer als auch das Risiko von Fehlern und Inkonsistenzen in den Dokumenten.

\textbf{Vermittlung oder Integration von Finanz- und Versicherungsdienstleistungen:} \\
Die Finanzierung und Absicherung von unternehmerischen Risiken stellt für viele Gründer eine zentrale Herausforderung dar.
Die Integration direkter Schnittstellen zu Banken oder ein Angebot der Finanzierung durch die hauseigene Bank ermöglicht eine beschleunigte und vereinfachte Kreditvergabe.
Durch die Übermittlung relevanter Daten aus dem Portal an die Banken können automatisierte Bonitätsprüfungen und Kreditentscheidungen ermöglicht werden. Dies verkürzt den Prozess der Kreditbeantragung erheblich und bietet Gründern schneller Planungssicherheit.
Dasselbe Prinzip kann auch auf Versicherungsprodukte angewendet werden.

\textbf{Berücksichtigung unterschiedlicher Rechtssysteme:} \\
Die langfristige Vision des Projekts ist die Integration in unterschiedliche Rechtssysteme, dies ermöglicht sowohl die Erschließung von neuen Märkten als auch die Möglichkeit für internationale Unternehmen die administrativen Aufwände zu bündeln und zentral zu orchestrieren.
Zwei zentrale negative Standortfaktoren Europas sind die sprachliche und rechtliche Heterogenität. Eine fehlende europäische Rechtsform für Unternehmen erschwert Startups schnelle Skalierung, wie sie in den USA und in China möglich sind.
Das Projekt könnte diese Hürden überwindbarer machen. Allerdings ist dies ein hochkomplexes Unterfangen und bedarf einer langfristigen Entwicklung. Voraussetzung sind eine umfassende Analyse der jeweiligen Rechtssysteme, eine flexible und skalierbare Architektur des Portals und die kontinuierliche Zusammenarbeit mit internationalen Rechtsexperten und Institutionen.


