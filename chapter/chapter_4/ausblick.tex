Durch die zeitlich begrenzte Natur dieses Projekts war es nicht möglich, alle gewünschten Funktionen im Rahmen des Projekts umzusetzen. Dennoch möchten wir drei Erweiterungen für ein zukünftiges Projekt hier erwähnen.

\textbf{Monitoring}

Zu einer Webseite, insbesondere einem Webshop, fallen viele nützliche Informationen an. Diese können aus unterschiedlichen Quellen kommen, wie z.B. aus Serverlogs oder von Shopify (Warenkorbinformationen). Da das Admin Center bereits eine zentrale Stelle für die Verwaltung aller Webseiten darstellt, ist es durchaus sinnvoll, alle Traffic-Informationen zu den Webseiten auch dort zentral abzubilden. Darüber hinaus könnte man dort noch weitere, typische Nutzerinteraktions-Metriken abbilden, wie z.B. die Absprungrate, die durchschnittliche Verweildauer oder die Scroll-Rate. Außerdem wäre es auch durchaus sinnvoll, weitere technische und performancerelevante Metriken abzubilden.

\textbf{Upgrade einer Webseite}

Das hier vorgestellte Konzept basiert darauf, möglichst schnell neue Webseiten bereitzustellen. Diese damit erstellten Webseiten sind im ersten Schritt immer erstmal Unterseiten der Hauptseite www.rf-innotrade.de, passend zur geschäftlichen Abbildung der Marken (welche als nicht eigenständige Entitäten unterhalb der RF InnoTrade angesiedelt sind). Sollte eine dieser neu erstellten Marken Erfolg haben und eine bestimmte Umsatzgrenze überschreiten, wird diese zu einer geschäftlich und rechtlich eigenständigen Entität umgewandelt. Dieser Schritt soll dann auch im Admin Center durchführbar sein in Form eines „Upgrades". Damit einher geht dann zum Beispiel das Vergeben einer komplett eigenständigen URL und, je nach Traffic auf der Webseite, auch das Auslagern des Containers auf einen eigenen Server.

\textbf{Integration von weiteren Systemen}

Um den Admin Center von RF InnoTrade noch umfassender bzw. vielseitiger zu gestalten, ist es eine Überlegung wert, in gewissem Maße andere benutzte Systeme mit in den Admin Center einzubinden. Das kann entweder durch eine simple Verknüpfung einer URL als Absprungpunkt in ein anderes System abgebildet werden (bspw. Absprung in ein Buchhaltungsprogramm) oder durch die tatsächliche Integration von Funktionen aus anderen Systemen. Ein Beispiel für Letzteres ist die Nutzung der Shopify API. Hiermit könnten beispielsweise die pro Marke vertriebenen Produkte direkt im Admin Center angezeigt werden mit der Option, kleinere Anpassungen an den Produkten vorzunehmen (z.B. Änderung des Preises). Auch das zentrale Steuern von Versandoptionen, Rabattaktionen usw. könnte dann über den Admin Center erfolgen.