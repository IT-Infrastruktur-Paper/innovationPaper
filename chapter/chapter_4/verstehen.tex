Die Gründung von Unternehmen und im weiteren Sinne das Anmelden eines Markenrechts in Deutschland, ist auch über mögliche Online-Kanäle ein nicht unerheblich aufwendiger Prozess\vglfootcite[]{oa_firma_nodate}.\\
Gleichzeitig ist die bürokratische und verwaltungstechnische Belastung der Unternehmen enorm und weiterhin steigend. Daran anschließend ist auch die fehlende Digitalisierung im Bereich der Verwaltung innerhalb von Deutschland eine weitere Hürde, die bei der Gründung eines Unternehmens hinzu kommt\vglfootcite[S. 13 ff.]{wohlrabe_burokratie_2024}.\\
In diesem Kontext ist uns aufgefallen, dass der ohnehin schon komplizierte Gründungsprozess, nicht einfach und schnell bewältigt werden kann – schon gar nicht digital. Zwar gibt es die Möglichkeit gewisse Behördengänge digital zu bewältigen, der Gesamtprozess bleibt dabei aber in viele Teilschritte zerstückelt, undurchsichtig und aufwendig.\\
Einige Mitglieder unserer Gruppe sind bisher kaum oder gar nicht in Berührung mit der Gründung von Existenzen gekommen und schrecken schon allein auf Grund des hohen Verwaltungsaufwands vor der Gründung eines Unternehmens zurück. Andere wiederum haben schon eigene Unternehmen gegründet und können den komplizierten und kaum digital zu bewältigenden Verwaltungsaufwand bestätigen.\\
Im anzuwendenen Beispiel der RFInnotrade, ist es aus sicht des Gründers, welcher im Laufe des design thinking Prozesses auch als Kunde dargestellt wird, ein erheblicher Aufwand und gleichzeitg ein großer einschränkender Faktor bei der Skalierung des eigenen Unternehmens. Gute Ideen können nicht schnell und einfach umgesetzt werden, sodass diese im Zweifel sogar verloren gehen.\\
Zwar spielen für den Erfolg der Idee auch andere Faktoren wie Geld und in erster Linie die Qualität des Produkts eine Rolle, jedoch sollte das Scheitern an den Gründungsprozessen nicht die finale Hürde darstellen.\\
Es fehlt also eine konkrete und anwendbare Möglichkeit, die Unternehmensgründung einfach, automatisch und möglichst digital umzusetzen. Dadurch fehlt es auch der RFInnotrade während der Gründung von weiteren Subunternehmen, Marken und Submarken, an einer einfachen Dokumentation und Verwaltungsmöglichkeit der laufenden Gründungsprozesse.\\