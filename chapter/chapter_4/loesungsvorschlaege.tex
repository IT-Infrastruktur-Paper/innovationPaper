In diesem Schritt wollten wir unsere verschiedenen Ideen in Form eines Brainstormings sammeln, bewerten und weiterentwickeln.
Im Laufe des Brainstormings ist allerdings recht schnell klar geworden, dass wir uns alle auf eine Umsetzung in Form einer Webseite fokussieren wollen und diese Variante auch als vielversprechende Möglichkeit einschätzen.
In Folge dessen haben wir uns also mehr auf die Weiterentwicklung der Idee und Umsetzung der einzelnen Kundenbedürfnisse fokussiert.
Das Grundkonzept besteht aus einem Portal, welches in Form einer Webseite umgesetzt wird. Ziel soll es sein, die vielen verschiedenen Anlaufstellen und Prozessschritte, zentral über das Portal zu bündeln und dabei alle genannten Kundenbedürfnisse möglichst digital umzusetzen.
Über das Portal soll der Kunde zunächst Schritt für Schritt an den Gründungsprozess und die damit verbundenen Teilschritte herangeführt werden. Voraussetzung ist, dass der Kunde die Idee für sein Unternehmen bereits ausgearbeitet hat und nun die konkrete Gründung in Angriff nimmt. Die notwendigen Schritte zur Unternehmensgründung, sollen ihm sowohl strukturiert und ansprechend visualisiert als auch erläutert werden. Für eine Umfirmierung soll das gleiche Konzept, mit angepassten Schritten verwendet werden. Die Prozesse der zu durchlaufenden Instanzen selber, können nicht direkt beeinflusst werden. Da wir aber trotzdem die Wartezeiten verkürzen und Rückfragen verringern möchten, garantieren wir auf Basis gründlichster Recherche für eine hohe Datenqualität und möglichste Vollständigkeit. Je länger und je mehr Kunden das Portal nutzen, desto größer werden automatisch die Erfahrungswerte und damit auch die Qualität der übermittelten Anträge.
Der digitale Identifikationsvorgang kann und soll über die digitale Ausweisfunktion unmittelbar im Portal integriert oder alternativ über ein bereits bestehendes Identifikationssystem vorgenommen werden. Zur digitalen Zahlungsabwicklung können herkömmliche Zahlungsmethoden in Anspruch genommen werden. Das Portal soll dabei als Factoring-Anbieter zwischen Kunde und Behörde dienen. Das bietet der Behörde und weiteren Instanzen eine Zahlungssicherheit und der Kunde erhält eine möglichst große Auswahl an Zahlungsmöglichkeiten.
Die Umsetzung zur Anzeige des aktuellen Prozessstatus, soll ebenfalls über das Portal und zu den jeweiligen Teilprozessen umgesetzt werden. Der Kunde soll wissen, ob seine Steuer-ID bereits beantragt wurde, gerade vom Finanzamt erstellt wird und benachrichtigt werden, falls Unterlagen nachgereicht werden müssen.
Um überhaupt die Anbindung an die verschiedenen Behörden und weiteren Stellen und die damit schnelle und unkomplizierte Abwicklung aller Prozesse zu ermöglichen, muss das Hauptfeature in die Seite integriert werden.
Auch an dieser Stelle des Design Thinking Prozesses, ist beim Brainstorming schnell klar geworden, dass dies über digitale Schnittstellen zu den Behörden und weiteren Instanzen bewältigt werden muss, die so weit wie möglich automatisiert verarbeitet werden können. Hier gab es ebenfalls eine einheitliche Vorstellung der Gruppe zur technischen Umsetzung. Es müssen sogenannte Application Programming Interfaces, kurz API, geschaffen werden. Mithilfe dieser Schnittstellen kann ein standardisierter Kommunikationsweg geschaffen werden, über den die Daten zwischen dem Portal und den Instanzen ausgetauscht werden können. Über diesen Kommunikationsweg werden dann auch die zuvor genannten Funktionen bedient.