In diesem Schritt wollten wir unsere verschiedenen Ideen in Form eines Brainstormings sammeln, bewerten und weiterentwickeln.\\
Im Laufe des Brainstormings ist allerdings recht schnell klar geworden, dass wir uns alle auf eine Umsetzung in Form einer Webseite fokussieren wollen und diese Variante auch als vielversprechende Möglichkeit einschätzen, da die bereits gegebene Struktur eine passende Plattform bietet.\\
In Folge dessen haben wir uns also mehr auf die Weiterentwicklung der Idee und Umsetzung der einzelnen Kundenbedürfnisse fokussiert und in mehreren Runden verschiedene Ideen und Aspekte eingebracht.\\
Das Grundkonzept besteht aus einem Portal, welches in Form einer Webseite umgesetzt wird. Ziel soll es sein, die vielen verschiedenen Anlaufstellen und Prozessschritte, zentral über das Portal zu bündeln und dabei alle genannten Kundenbedürfnisse möglichst digital umzusetzen.\\
Über das Portal soll der Kunde zunächst, Schritt für Schritt, an den Gründungsprozess und die damit verbundenen Teilschritte herangeführt werden. Voraussetzung ist, dass der Kunde die Idee für sein Unternehmen oder seine Marke bereits ausgearbeitet hat und nun die konkrete Gründung oder Anmeldubg in Angriff nimmt. Die notwendigen Schritte zur Unternehmensgründung sollen ihm sowohl strukturiert und ansprechend visualisiert als auch erläutert werden. Für eine Umfirmierung soll das gleiche Konzept, mit angepassten Schritten verwendet werden.\\
Durch die gut ausgearbeitete Dokumentation der einzelnen Schritte, soll die Qualität der Anträge so weit angehoben werden, dass diese von den verschiedenen Instanzen möglichst unkompliziert und ohne weitere Rückfragen bearbeitet werden können.\\
Der digitale Identifikationsvorgang kann und soll über die digitale Ausweisfunktion unmittelbar im Portal integriert oder alternativ über ein bereits bestehendes Identifikationssystem wie PostIdent vorgenommen werden können. Zur digitalen Zahlungsabwicklung soll eine Schnittstelle zur hauseigenen Bank in die Webseite integriert werden. Über diese soll dann bei Unternehmensgründung ein neues Geschäftskonto angelegt und zur Abbuchung aller Gebühren an die verschiedenen Instanzen übermittelt werden.\\
Die Umsetzung zur Anzeige des aktuellen Prozessstatus, soll ebenfalls über das Portal und zu den jeweiligen Teilprozessen umgesetzt werden. Der Kunde soll wissen, ob seine Steuer-ID bereits beantragt wurde, gerade vom Finanzamt erstellt wird und benachrichtigt werden, falls Unterlagen nachgereicht werden müssen. Gleiches gilt für die Kontoeröffnung oder ein Genehmigungsverfahren.\\
Um überhaupt erst die Anbindung an die verschiedenen Behörden sowie die weiteren Stellen und die damit schnelle und unkomplizierte Abwicklung aller Prozesse zu ermöglichen, müssen entsprechende Schnittstellen zu den jeweiligen Instanzen geschaffen werden. Die Schnittstellen müssen so gebaut werden, dass eine Kommunikation in beide Richtungen möglich ist, um den aktuellen Prozesstatus sichtbar zu machen und um Dateien übermitteln sowie Anfordern zu können.\\
Um den Kunden bei der Findung einer passenden Rechtsform oder bei der rechtlichen Recherche zur Anmeldung einer Marke zu unterstützen, soll eine KI-Funktion in das Portal integriert werden. Diese soll dem Kunden anhand von gezielten Fragen und Antworten, die passende Rechtsform vorschlagen und rechtliche Zweifel bei der Markenanmeldung ausräumen.\\