Der Design Thinking Prozess ist eine systematische Innovationsmethode, bestehend aus fünf Phasen. Der Fokus des Design Thinking Prozess liegt dabei auf den Bedürfnissen des Menschen. Entsprechend dieser Bedürfnisse sollen die Lösungen darüber hinaus möglichst innovativ und technisch machbar sein\vglfootcite[S. 5 ff.]{vetterli_innovationsmethode_2012}.\\
Zu Beginn definiert das Design Team die Problemstellung und versucht darüber hinaus das gegebene Umfeld rund um das Problem, sowie wichtige Einflussfaktoren zu erfassen und zu verstehen. Dies ist die Phase des Verstehens.\\
Im zweiten Schritt folgt die Definitionsphase. In dieser Phase werden die Kundenbedürfnisse analysiert und erfasst. Dies kann durch direkte Kundenansprache, Visualisierungen oder durch Beobachtungen im Gesamtkontext realisiert werden.\\
Danach folgt der dritte Schritt, die sogenannte Entwurfsphase. In diesem dritten Schrtitt werden konkrete Lösungsvorschläge über einen kreativen Prozesse, wie zum Beispiel Brainstorming, erstellt. Hier kann jedes Gruppenmitglied uneingeschränkt eigene Lösungsvorschläge einbringen.\\
In Schritt vier folgt das Prototyping. Hierbei werden die in Schritt 3 gesammelten Ideen in anfassbare Prototypen umgesetzt. Die Abstraktionsstufen dieser Prototypen kann je nach Projektphase entsprechend variieren.\\
Im fünften und letzten Schritt des Design Thinking Prozesses erfolgt das Testing. Die zuvor erstellten Prototypen werden von realen Nutzern getestet, wobei Fehler ganz bewusst erlaubt und sogar gewünscht sind. Diese Fehler können aufgenommen, daraus gelernt und der Prototyp verbessert werden.\\
Die einzelnen Phasen, können jederzeit mehrfach durchlaufen werden, um Stück für Stück die Lösung des Problems zu optimieren. Im Idealfall setzen sich die Mitglieder des Design Thinking Prozesses aus verschiedenen Fachrichtungen zusammen, um unterschiedliche Perspektiven auf die Problemstellung einzubringen. Gleichzeitig werden Prototypen möglichst schnell erstellt, um früh erste Lösungsansätze testen zu können. Radikale und ungewöhnliche Lösungsansätze sind während des gesamten Prozesses ebenfalls gewünscht\vglfootcite[S. 8 ff.]{vetterli_innovationsmethode_2012}.\\
In den Folgenden Unterkapiteln gehen wir auf unsere Umsetzung der einzelnen Phasen ein.