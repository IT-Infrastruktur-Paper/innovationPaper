Der Trend geht aktuell dahin, eigene Ideen möglichst schnell und einfach umzusetzen und zum Beispiel in Form eines Start-Ups an den Markt zu gehen. Dieser Schritt erscheint auf Grund der in der ersten Phase analysierten Probleme als große Hürde. In diesem Kontext haben wir, in Form einer Mindmap und über mehrere Sessions hinweg, konkrete Kundenbedürfnisse zusammengetragen.
Ein Bedürfnis der Kunden ist nach unserer Analyse, ein einfacher Überblick über alle für den Gründungsprozess notwendigen prozessualen Schritte. Dazu gehört auch der Prozess der Umfirmierung eines bereits bestehenden Unternehmens.
Ein weiteres Bedürfnis der Kunden ist die Möglichkeit, den gesamten Prozess digital und aus einer Hand heraus abzuwickeln. Gemeint ist damit der Wunsch nach einer Umsetzung über eine einzige Anlaufstelle, da zurzeit viele verschiedene Instanzen einzeln und nacheinander aufgesucht werden müssen. Außerdem sollen Anträge und Genehmigungsverfahren, möglicht schnell und ohne Rückfragen abgewickelt werden können. Damit Verbunden ist auch eine Reduzierung der Rückfragen auf Grund fehlender oder falsch eingereichter Unterlagen.
Im Kontext der Digitalisierung ist es auch notwendig, die personenbezogene Identifizierung vollständig digital durchführen zu können. Ebenfalls sollte es digital möglich sein, den aktuellen Status der Anträge und Genehmigungsverfahren abfragen zu können und die Bezahlung des Gesamtvorgangs digital abwickeln zu können. Ein weiteres digitales Feature, welches für Kunden in Frage kommt, ist die einfache Erstellung einer Unternehmenswebseite mit Grundfunktionen als erstes Aushängeschild.
Als weiteres Bedürfnis haben wir die Unterstützung bei der Auswahl zur passenden Rechtsform und bei der rechtlichen Recherche zur Anmeldung einer Marke ausgemacht.
In einer späteren Wiederholung dieser Phase wurde nochmals hervorgehoben, dass es am Ende möglichst wenige oder nur ein Klick bis zum Ziel sein soll. Danach sollen sich die Funktionen des Portals ausrichten.