Im Folgenden werde ich nochmal kurz und knapp die Vorteile unserer gewählten Struktur und Technologien hervorheben.

Ein Vorteil von der von uns gewählten Struktur, ist die Modularität und Wiederverwendbarkeit. Diese wird vor allem durch die Trennung von Front- und Backend erreicht. Hinzu kommt die Verwendung einzelner Komponenten innerhalb von React, wodurch Wartung und Erweiterung der Webseite verbessert werden\footcite{acharya_15_2023}.

Darüber hinaus ist, ebenfalls durch die Trennung von Front- und Backend, aber auch durch die Nutzung von Strapi als headless CMS, eine hohe Flexibilität bei der Entwicklung gegeben. Dadurch kann jederzeit auf äußere Einflüsse wie neue Technologien oder Kundenanforderungen eingegangen werden\footcite{autor_warum_nodate}.

Die Verwendung von APIs für den Datenfluss zwischen Front- und Backend, bietet den Vorteil, dass dieser effizient und standardisiert ablaufen kann. Strapi stellt die standardisierten APIs bereit und React greift darüber auf die benötigten und wiederverwendbaren Daten zu\footcite{autor_headless-cms_nodate}.

Strapi bietet zudem einen großen Vorteil in Sachen Sicherheit. Die integrierten Sicherheitsfunktionen in Strapi schützen das Backend und im Frontend können unabhängig davon weitere Sicherheitsmaßnahmen gezielt getroffen werden.
