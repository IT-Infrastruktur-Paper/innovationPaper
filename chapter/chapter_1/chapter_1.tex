\newpage

\section{Einleitung} \label{einleitung}
Diese Hausarbeit wird im Namen der Firma \textit{RF InnoTrade GmbH} durchgeführt. Die Firma ist ein mittelständisches Unternehmen, welches sich auf das vermarkten und vertreiben von Trend- und Lifestyleprodukten spezialisiert hat. Die \textit{RF InnoTrade GmbH} ist ein reines E-Commerce Unternehmen und hat keine Ladenlokale. Die Führungsetage möchte den Umsatz steigern und hat dafür eine neue Diversifizierungsstrategie entwickelt. Die Strategie sieht eine maximale Diversifizierung der Marken und Produkte vor um auf die stetig und schnell wechselnden Trends, welche Hauptsächlich durch Social Media Plattformen wie Instagram, TikTok und Pinterest beeinflusst werden, reagieren zu können. Dafür ist es notwendig so schnell wie möglich neue Webseiten, neue Marken oder sogar neue Firmen zu launchen.
Eine Plattform dafür wurde bereits geschaffen. Mit dieser Plattform ist die Firma \textit{RF InnoTrade GmbH} allerdings nur in der Lage sehr schnell neue Webseiten mit optional integriertem Webshop ins leben zu rufen. Um die Firmstruktur auszubauen und die Diversifizierungsstrategie umzusetzen, ist es notwendig auch neue Firmen zu gründen und gegebenenfalls Marken anzumelden. Jeder Gründungs-und Anmeldeprozess geht jedoch mit viel behördlicher Komplexität, langen Wartezeiten und von Prozess zu Prozess unterschiedlichen Status bzw. Ständen des Prozessfortschritts einher. Um hier die Stratgie bestmöglich umzusetzen muss das Portal angepasst werden, einerseits um den Prozess zu beschleunigen und damit Kosten zu sparen, andererseits um den Überblick zu bewahren.
Daher hat die Firma \textit{RF InnoTrade GmbH} ein junges Team damit beauftragt eine Lösung für diese Problem zu finden. Der Prozess der Ideenfindung und Problemlösung wird in dieser Arbeit beschrieben.