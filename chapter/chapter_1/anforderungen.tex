Die Entwicklung moderner Webanwendungen stellt Entwickler vor verschiedene Herausforderungen. Dabei geht es nicht nur um die reine Funktionalität, wie zum Beispiel das Umsetzen eines funktionierenden Webshops, der Webseite an sich. Bei der Entwicklung einer Webseite geht es darüber hinaus auch um die Performance, Benutzerfreundlichkeit, Flexibilität und Skalierbarkeit. Dabei können die unterschiedlichen Aspekte, je nach Anforderung, unterschiedlich stark gewichtet werden, wobei die reine Performance und die Benutzerfreundlichkeit der Webseite für den Nutzer meistens von hoher Bedeutung sind.
Die Performance spielt eine große Rolle, da sie den Erfolg der Webseite beeinflussen kann. Nutzer erwarten kurze Ladezeiten und einen einwandfreien Ablauf der Funktionalität\footcite{deinhard_uberblick_2024}.
Die Benutzerfreundlichkeit einer Webseite ist deshalb so wichtig, da sie überhaupt erst die Akzeptanz zur Nutzung einer Webseite beeinflusst. Eine Webseite muss grundsätzlich gut lesbar, navigierbar und anpassbar für viele Verschiedene Geräte und Browser sein\footcite{siever_responsive_2024}.
Auch die Flexibilität einer Webseite spielt eine übergeordnete Rolle. Sowohl ständig wechselnde Geschäftsanforderungen als auch fortschreitende Technologien setzen voraus, dass Webseiten in der Lage sein müssen auf diese Veränderungen beispielsweise mithilfe von modularen Technologien wie React reagieren zu können\footcite{wiedermayer_webentwicklung_2024}.
Zuletzt ist die Skalierbarkeit einer Webseite insofern wichtig, dass durch unterschiedliche Nutzerzahlen, der Ressourcenbedarf einer Webseite stark schwanken und dynamisch anpassbar sein sollte\footcite{annacherniavska_erstellung_2023}.
Um die genannten Anforderungen möglichst gut umzusetzen, ist es essenziell wichtig, Frontend und Backend grundsätzlich voneinander zu trennen. Die modulare Vorgehensweise wird hierdurch noch einmal bestätigt und ermöglicht eine flexible Vorgehensweise in der Entwicklung. Die Trennung von Frontend und Backend ist außerdem hilfreich, damit die Entwicklung der beiden Komponenten parallel und unabhängig voneinander erfolgen kann\footcite{hort_warum_2018}. Dadurch kann bei der Entwicklung Zeit gespart werden und es können schneller Funktionalitäten der Webseite fertiggestellt und getestet oder veröffentlicht werden.