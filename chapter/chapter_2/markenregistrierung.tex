Die Anmeldung oder Registrierung einer Marke in Deutschland ist behördlich gesehen kein sonderlich komplexer Prozess. Es gibt lediglich eine Instanz, die in Deutschland für den Schutz geistigen Eigentums zuständig ist: das Deutsche Patent- und Markenamt (DPMA). Beim Prozess der Anmeldung muss lediglich angegeben werden, um welche Marke es sich handelt, in welcher Form die Marke eingereicht wird (Wort-, Bild- oder Wort-Bildmarke) sowie zu welcher Waren- und Dienstleistungsklasse (Nizza-Klasse) die Marke gehört. 

Mit diesen Informationen kann der Antrag auf Markenregistrierung entweder analog oder digital eingereicht werden. Danach muss innerhalb von 3 Monaten die Bearbeitungsgebühr in Höhe von 300 € bei analogen und 290 € bei digitalen Anträgen entrichtet werden und der Prozess ist abgeschlossen. Nach einer gewissen Zeit meldet sich das DPMA beim Antragsteller mit dem Ergebnis der Überprüfung zurück. Wichtig zu wissen ist allerdings, dass das DPMA hierbei nur auf die sogenannten "absoluten Schutzhindernisse" geprüft hat, im Grunde ob die Eintragung der Marke gegen bestimmte Regeln verstößt. Eine Marke verstößt unter folgenden Umständen gegen die Regeln und ist nicht schutzfähig \vglfootcite[]{oa_markenrecht_nodate}: 

\begin{itemize}
    \item Fehlende Unterscheidungskraft (z.B. "marktfrisch" für Lebensmittel, nicht schutzfähig)
    \item Freihaltebedürfnis (z.B. "Dreier" für KFZs (BMW) ist schutzfähig, "Cabrio" für KFZs wiederum nicht)
    \item Gattungsangaben/übliche Bezeichnungen ("Benzin" als Marke für Kraftstoff ist nicht schutzfähig, als Marke für Kleidung wiederum schon)
    \item Täuschende Zeichen
    \item Ordnungs- oder sittenwidrige Zeichen
    \item Hoheitszeichen
    \item Amtliche Prüfzeichen
    \item Kennzeichen internationaler Organisationen
\end{itemize}

Lautet das Ergebnis des DPMA, dass eine Eintragung der Marke nicht möglich ist, weil sie nicht schutzfähig ist, muss ein neuer Antrag gestellt werden, was gleich hohe Kosten erzeugt wie der ursprüngliche Antrag. Um die behördlichen Prozesskosten so gering wie möglich zu halten, ist es ratsam, den Antrag auf eben jene Kriterien akribisch zu prüfen. Nach der Sicherstellung der allgemeinen Schutzfähigkeit der Marke kann nun der Prozess der eigentlichen Markenrecherche beginnen. Diese verhindert das Verletzen älterer Rechte Dritter, was vom DPMA nicht geprüft wird \vglfootcite[]{oa_markenrecht_nodate}. Die Markenrecherche kann und sollte an verschiedenen Orten stattfinden, angefangen bei einer einfachen Recherche im Web. Hier kann auf verschiedene Quellen zurückgegriffen werden wie z.B.:

\begin{itemize}
    \item verschiedene Suchmaschinen (Google, Bing, etc.)
    \item Domain Name Suchen (z.B. NameCheap oder Instant Domain Search)
    \item Social Media (Facebook, Instagram, TikTok, LinkedIn, etc.)
\end{itemize}
    
Sollte diese Form der Vorabsuche keine Ergebnisse hervorgebracht haben, so kann mit der Identitäts- und darauf mit der Ähnlichkeitsrecherche fortgefahren werden \vglfootcite[]{plutte_markenrecherche_2015}. Bei der Identitätsrecherche werden die Datenbanken und Register nach dem exakten Wortlaut identischer älterer Marken und deren Markenklassen durchsucht. Die Ähnlichkeitsrecherche ist aufwändiger, da in dieser Recherche nach Marken gesucht wird, die der geplanten Marke im Wortlaut, Schriftbild oder Design ähneln \vglfootcite[]{plutte_markenrecherche_2015}. Für beide Fälle kann auf verschiedene Quellen zurückgegriffen werden \vglfootcite[]{oa_dpma_nodate-1}: 

\begin{itemize}
    \item DPMAregister
    \item eSearch plus
    \item Madrid Monitor
    \item TMview (Register des EUIPO)
    \item Global Brand Database (Register der WIPO)
\end{itemize}

Erst wenn all diese Recherchen akribisch durchgeführt wurden, ist es zu empfehlen, einen Antrag auf Markenregistrierung zu stellen. Ist die Genehmigung seitens des DPMA erteilt, so beginnt die 3-Monatige Widerufsfrist in der Dritte Wideruf einlegen dürfen. 
Wie zu sehen ist, liegt die Schwierigkeit bei der Registrierung einer Marke hier ausnahmsweise nicht im behördlichen Aufwand, sondern in der Markenrecherche \vglfootcite[]{wolking_-_2007}. Wird diese nicht in ausreichendem Umfang gemacht, läuft das Unternehmen Gefahr, hohe Prozesskosten zu verursachen und den Prozess der Markenregistrierung in die Länge zu ziehen.