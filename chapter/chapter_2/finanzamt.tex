Nach der erfolgreichen Anmeldung eines Gewerbes durch das Gewerbeamt, muss der Gründer seine Unternehmung steuerlich erfassen, um Waren und Dienstleistungen legal in Deutschland anbieten zu können.
 Verantwortlich für die steuerliche Erfassung der Unternehmung ist das zuständige Finanzamt, welches eine Steuernummer vergibt. Diese ist unerlässlich für die Ausstellung von Rechnungen sowie zur Abgabe von Steuererklärungen. \vglfootcite[]{oa_formular-management-system_nodate}
 Zur zentralen Abwicklung von Steuerangelegenheiten dient die vom Finanzamt implementierte Plattform \"ELSTER".
 Diese kann sowohl für die steuerliche Erfassung von Unternehmen, als auch für die Einreichung einer Steuererklärung dienen. 
 Über die Plattform ELSTER kann die steuerliche Erfassung mit dem „Fragebogen zur steuerlichen Erfassung“ komplett digital an das zuständige Finanzamt übermittelt werden, Voraussetzung ist die vorherige Registrierung eines Benutzeraccounts mit einer E-Mail-Adresse. 
 Die Zustellung der Steuernummer erfolgt jedoch trotzdem noch analog per Post. \vglfootcite[]{oa_elster_nodate}
