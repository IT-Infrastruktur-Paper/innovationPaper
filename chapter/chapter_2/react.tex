React ist das von uns bevorzugte Frontend. Im Folgenden werden wir erklären, was React ist und wie es grundsätzlich aufgebaut ist.
React wurde von Facebook entwickelt und ist eine leistungsstarke Open-Source-Bibliothek, basierend auf JavaScript. Dabei ist React auf die Erstellung von Benutzeroberflächen, also auf das sogenannte Frontend, spezialisiert\footcite{acharya_15_2023}. Darüber hinaus ist React eine komponentenbasierte Bibliothek, welche einen effizienten und strukturierten Ansatz zur Gestaltung von interaktiven Benutzeroberflächen bietet\footcite{p_was_2024}. 
Aus diesen Gründen eignet sich die Bibliothek hervorragend für komplexe Benutzeroberflächen, welche eben aus diesen kleinen und modularen Komponenten bestehen. Für die Benutzer bedeutet der Einsatz von React anwenderfreundliche Oberflächen und für die Entwickler eine verbesserte Code-Organisation, was unter anderem zu einer erleichterten Wartung führt\footcite{autor_komponentenbasierte_nodate}.
Zudem verwendet React ein virtuelles Document Object Model (DOM). Das virtuelle DOM ist ähnlich zum tatsächlichen DOM. Es ist allerdings wesentlich leichtgewichtiger und wird als JavaScript-Objekt dargestellt. Auf Grund der Darstellung als JavaScript-Objekt werden lediglich die wirklich notwendigen Änderungen am tatsächlichen DOM durchgeführt, wodurch wiederum die Performance optimiert wird\footcite{autor_mastering_2023}.
React Komponenten verwenden grundsätzlich zwei verschiedene Konzepte zur Datenverwaltung. Diese beiden Konzepte nennen sich States und Props und arbeiten im Zusammenspiel, um dynamische und interaktive Benutzeroberflächen zu erstellen\footcite{autor_react_nodate}.
States repräsentieren den internen Zustand einer Komponente und können sich im Laufe der Zeit verändern. Sie werden von der React Komponente selber verwaltet und können sowohl durch Benutzeraktionen als auch durch interne Ereignisse verändert werden\footcite{autor_state_nodate-1}. Dies ermöglicht überhaupt erst die interaktive Nutzung der Webseite im laufenden Betrieb.
Props dagegen sind als Kurzform für Properties, zu Deutsch Eigenschaften, zu verstehen und sind unveränderlich. Sie dienen dazu, bestimmte Funktionen und Daten von einer sogenannten Elternkomponente an andere Komponenten zu übertragen\footcite{autor_state_nodate}.