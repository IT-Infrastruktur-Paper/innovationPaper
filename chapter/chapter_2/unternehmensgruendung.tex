Die Unternehmensgründung in Deutschland ist ein komplexes Unterfangen mit vielen, teils manuellen, aber dennoch verpflichtenden Schritten. Viele dieser Schritte sind über verschiedene Instanzen verteilt, können nur teilweise digital abgewickelt werden und fordern potenziellen Gründern einiges an Durchhaltevermögen ab. Weitere Faktoren, die das Gründen eines Unternehmens erschweren, sind die teils dezentralisierte Verwaltung von Ämtern (z.B. Gewerbe- und Finanzamt, kann im Grad der Digitalisierung und Geschwindigkeit des Abarbeitens von Aufträgen stark schwanken je nach Region) und die teils sehr langwierigen Prozesse (z.B. das Warten auf eine Umsatzsteueridentifikationsnummer beim örtlichen Finanzamt). Eine weitere Komplexitätsstufe sind die je nach Rechtsform des anzumeldenden Unternehmens unterschiedlichen Prozessschritte, die durchlaufen werden müssen. Aus diesem Grund wird im folgenden Abschnitt unterschieden zwischen der Gründung einer Personengesellschaft (inkl. Einzelunternehmen) und einer Kapitalgesellschaft. Was jedoch für beide Formen der Gründung gilt, ist, dass vorab einige Rahmenparameter klar definiert sein sollten. Dies umfasst unter anderem: 

\begin{itemize}
    \item Geschäftsführer- / Gesellschafterverhältnisse
    \item Name des Unternehmens
    \item Sitz des Unternehmens
    \item Art des Unternehmens / Rechtsform
    \item Tätigkeitskategorie + detaillierte Tätigkeitsbeschreibung
    \item (Start-) Finanzierung
    \item Erste (grobe) Hochrechnung Umsatz + Gewinne erstes Geschäftsjahr
\end{itemize} 

\textbf{Personengesellschaft}

Nachdem die vorher genannten Rahmenparameter klar sind, kann der offizielle Prozess starten. Bei Personengesellschaften beginnt dieser Prozess beim örtlichen Gewerbeamt. Hier wird die sogenannte Gewerbeanmeldung vollzogen. Die erste Anlaufstelle dafür ist immer das jeweils zuständige Gewerbeamt, in manchen Fällen kann auch von einer örtlichen Instanz an eine zentrale, übergeordnete Instanz verwiesen werden. Am Beispiel der Stadt Troisdorf ist zu sehen, dass der analoge Weg über das Ausfüllen eines Formulars noch über das Gewerbeamt Troisdorf selbst läuft. Besteht der Wunsch, das Formular online auszufüllen und zu übermitteln, wird an das Wirtschafts-Service-Portal NRW (WSP.NRW) verwiesen. Hier kann der Antrag in Gänze online ausgefüllt und versandt werden. Das Gewerbeamt meldet sich zurück mit eventuellen Rückfragen oder mit der Bestätigung der Gewerbeanmeldung. Nach erfolgreicher Gewerbeanmeldung ist der nächste Schritt die Meldung beim örtlichen Finanzamt. Hier muss der Fragebogen zur steuerlichen Erfassung ausgefüllt werden, was allerdings seit dem 01.01.2021 ausschließlich über das Webportal ELSTER erfolgen darf \footcite{oa_formular-management-system_nodate}. Diese steuerliche Erfassung muss bis spätestens 1 Monat nach der Gewerbeanmeldung passiert sein \footcite{oa_fragebogen_nodate}. Nach erfolgreicher Übermittlung und anschließender Genehmigung und Erteilung einer Steuernummer (ggfs. auch der Umsatzsteueridentifikationsnummer) seitens des Finanzamts sollte der nächste Schritt der zur Bank sein. Ein Geschäftskonto ist bei Personengesellschaften nicht verpflichtend, aus organisatorischen und buchhalterischen Gründen ist es allerdings ratsam, ein eigenes Konto für jedes Unternehmen anzulegen. Beim Großteil der etablierten Banken können Konten rein digital angelegt werden. Ein wichtiger, aber nicht zwingend verpflichtender Schritt ist die Eintragung ins Handelsregister. Die Verpflichtung zu diesem Schritt hängt bei Personengesellschaften von der Rechtsform des Unternehmens, des Eintragenden (eingetragener Kaufmann oder nicht) sowie der monetären Schwellenwerte ab (über 600.000€ Umsatz und 60.000€ Gewinn im Geschäftsjahr) \footcite{oa_kaufmannseigenschaft_nodate}. Dieser Schritt sollte wohl überlegt sein, da dieser immer über einen Notar geschieht und damit mit Kosten verbunden ist. Seit 2022 kann, muss dies aber nicht mehr physisch passieren, eine Videokonferenz mit Identifizierung über den Personalausweis mit eID-Funktion genügt \footcite{oa_firma_nodate}. 

\textbf{Kapitalgesellschaft}

Viele Schritte der Gründung einer Kapitalgesellschaft sind identisch zu denen der Gründung einer Personengesellschaft, mit dem Unterschied, dass die Reihenfolge eine andere ist. Da die Gründung einer Kapitalgesellschaft mit höheren Kosten verbunden ist als die einer Personengesellschaft, ist es ratsam, alle notwendigen Vorkehrungen im Vorhinein zu treffen. Neben der grundsätzlichen Startfinanzierung ist es auch ratsam, einen Business Plan zu erstellen. Um alle gesetzlichen Bedingungen zu erfüllen, sollten sich die Gründer außerdem vorab zusammensetzen und einen Gesellschaftsvertrag sowie ein Gründungsprotokoll verfassen \footcite{oa_gmbh-_nodate}. Haben die Gründer die eben genannten Grundlagen geschaffen, kann der erste offizielle Schritt eingeleitet werden. Dieser führt die Gründer zum Notar, welcher, wie vorhin bereits erwähnt, auch digital abgehalten werden kann. Der Notar beglaubigt die mitgebrachten Dokumente und bereitet die Anmeldung beim Handelsregister vor, welche dann auch von ihm durchgeführt wird. Ungleich zur Personengesellschaft ist die Anmeldung im Handelsregister bei Kapitalgesellschaften immer verpflichtend. Mit der Beglaubigung der Dokumente durch den Notar ist das Unternehmen bereits handlungsfähig, muss allerdings immer den Zusatz i.G. (in Gründung) mitführen. Damit der Notar das Unternehmen beim Handelsregister eintragen kann, ist jedoch ein weiterer Schritt durch die Gründer notwendig. Diese müssen sich ein Geschäftskonto bei einer Bank ihrer Wahl einrichten und dort das benötigte Stammkapital einzahlen. Die Höhe des Stammkapitals hängt von der gewählten Form der Kapitalgesellschaft ab (1€ bei UGs, 25.000€ bei GmbHs). Der Bank muss der beurkundete Gesellschaftsvertrag vorgelegt werden, dafür erhält der Gründer wiederum nach Einzahlung des Kapitals den notwendigen Nachweis über die Einzahlung. Mit diesem Nachweis geht es dann wieder zurück zum Notar, welcher die Anmeldung beim Handelsregister vollziehen kann. Mit erfolgreicher Eintragung ist die nun nicht mehr in Gründung befindliche Gesellschaft entstanden. Was bei der Personengesellschaft der erste Schritt war, folgt nun bei der Kapitalgesellschaft, der virtuelle Gang zum Gewerbeamt. Der Ablauf dessen ist identisch zu dem der Personengesellschaft. Auch die darauf folgende Meldung beim Finanzamt ist prozessual identisch zu der der Personengesellschaft. Der finale Schritt ist die Eintragung der wirtschaftlich Berechtigten Personen in das sogenannte Transparenzregister, eine Institution, die 2017 im Rahmen des Geldwäschegesetzes eingeführt wurde und der erhöhten Transparenz von juristischen Personen dient \footcite{oa_transparenzregister_nodate}. Die Anmeldung erfolgt ausschließlich digital. Darüber hinaus gibt es noch weitere Schritte, die notwendig sein können, basierend auf der Frage, ob das Unternehmen als Arbeitgeber auftritt und je nach Branche und Tätigkeit des Unternehmens. Dazu zählen unter anderem die Anmeldung bei der Berufsgenossenschaft, die Anmeldung bei der Agentur für Arbeit oder die Anmeldung bei der Handwerkskammer. Viele, aber nicht alle, der hier genannten Schritte treffen auch auf eine Ummeldung/Umfirmierung eines Unternehmens zu. Da die Details dieser Verfahren für diese Ausarbeitung eine geringere Relevanz haben, wurde von einer detaillierteren Ausarbeitung abgesehen.