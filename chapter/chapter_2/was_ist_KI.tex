Der Begriff der Künstlichen Intelligenz (KI) wird vom Bundesverband Informationswirtschaft, Telekommunikation und neue Medien (BITKOM) in einem Positionspapier wie folgt definiert. „Künstliche Intelligenz ist die Eigenschaft eines IT-Systems, „menschenähnliche“, intelligente Verhaltensweisen zu zeigen.“ Künstliche Intelligenz basiert auf den Technologien Machine Learning (ML) und Deep Learning (DL) und lässt sich in „schwache“ und „starke“ KI unterscheiden. Schwache KI ist nur darauf trainiert bestimmte Aufgaben auszuführen und macht den Großteil der heute verfügbaren KI-Lösungen aus. Starke KI ist ein theoretisches Konzept, bei dem eine Maschine und ein Mensch über gleichwertige Intelligenz verfügen würden. 