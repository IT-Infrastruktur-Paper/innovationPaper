Wenn das Unternehmen als Personengesellschaft (e.K, OHG, KG) oder als Kapitalgesellschaft (GmbH, AG) beim Gewerbeamt angemeldet worden ist, muss dieses ins Handelsregister eingetragen werden. 
Die Eintragung ins Handelsregister erfolgt zu rechtlichen Transparenzzwecken, dokumentiert die Unternehmensdaten und ist rechtlich unbedingt notwendig. \vglfootcite[]{oa_kaufmannseigenschaft_nodate}
Eine Handelsregisteranmeldung erfolgt in Jedem Fall über einen Notar der die einzutragenden Daten elektronisch an das Handelsregister weiterleitet. 
Um eine Anmeldung im Handelsregister komplett online durchzuführen, bietet die Bundesnotarkammer einen Onlineservice an. 
Der Anmeldungsprozess teilt sich in mehrere Abschnitte. Zuerst wird ein Formular zum Anliegen ausgefüllt und der Gründer muss sich mittels seiner eID identifizieren über welche eine qualifizierte elektronische Signatur generiert wird. 
Nachfolgend muss der Gründer einen Online-Termin mit einem Notar vereinbaren. In diesem wird der Urkundsentwurf durchgesprochen und anschließend rechtssicher unter Zuhilfenahme der qualifizierten elektronischen Signatur unterschrieben. \vglfootcite[]{oa_ablauf_nodate}
