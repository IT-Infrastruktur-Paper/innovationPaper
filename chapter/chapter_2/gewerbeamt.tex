Der erste Schritt zur Gründung eines Unternehmens ist die Anmeldung eines Gewerbes, die durch das zuständige Gewerbeamt erfolgt. 
 Dazu wird dem Gründer ein Gewerbeschein ausgestellt, der die Grundlage für die Ausübung einer unternehmerischen Tätigkeit darstellt. \vglfootcite[]{oa_unternehmensanmeldung_nodate}
 Obwohl es in Deutschland keine einheitliche, übergreifende Schnittstelle zur Gewerbeanmeldung gibt,  bieten einige Bundesländer Portale zur digitalen Anmeldung von Gewerben an. 
 In Nordrhein-Westfalen zum Beispiel kann die Gewerbeanmeldung über ein Onlineformular des Wirtschafts-Service-Portals NRW durchgeführt werden. \vglfootcite[]{oa_business-registration_nodate}
 Auch das Land Berlin stellt über das Service Portal Berlin eine komplett digitale Möglichkeit der Gewerbeanmeldung über ein Onlineformular bereit. \vglfootcite[]{oa_gewerbe_nodate}
 In Hessen erfolgt die Gewerbeanmeldung über die Dienstleistungsplattform des Landes Hessen. \vglfootcite[]{oa_information_nodate}
 Alle anderen deutschen Bundesländer stellen keine einheitliche Lösung zur Gewerbeanmeldung bereit, allerdings bieten einige Städte wie Potsdam oder Dresden ebenfalls einen Service zur Online-Gewerbeanmeldung an.

