
Bei der Gründung eines Unternehmens sind wie bereits beschrieben mehrere Behörden
involviert, die unterschiedliche Rollen im Gründungsprozess erfüllen.

Der erste Schritt der Unternehmensgründung ist die Anmeldung eines Gewerbes. Dieser erfolgt
durch das Gewerbeamt. Zur Anmeldung wird dem Gründer ein Gewerbeschein angemeldet, der die Grundlage
für die Ausübung einer unternehmerischen Tätigkeit darstellt.

Wenn das Unternehmen als Personengesellschaft (e.K, OHG, KG) oder als Kapitalgesellschaft (GmbH, AG)
beim Gewerbeamt angemeldet worden ist, muss dieses ins Handelsregister eingetragen werden. Die Eintragung 
ins Handelsregister erfolgt zu rechtlichen Transparenzzwecken, dokumentiert die Unternehmensdaten und ist rechtlich
unbedingt notwendig.

Nach der erfolgreichen Anmeldung eines Gewerbes durch das Gewerbeamt, muss der Gründer seine Unternehmung 
steuerlich erfassen, um Waren und Dienstleistungen legal in Deutschland anbieten zu können. Verantwortlich für
die steuerliche Erfassung der Unternehmung ist das zuständige Finanzamt, welches Dieses eine Steuernummer vergibt. 
Diese ist unerlässlich für die Ausstellung von Rechnungen sowie zur Abgabe von Steuererklärungen.

Weitergehend ist es für einige Unternehmen Sinnvoll, einer Kammer wie der Industrie- und Handelskammer (IHK) anzugehören.
Sofern der Abschluss der Mitgliedschaft nicht schon bei der Gewerbeanmeldung erfolgt ist, muss die Mitgliedschaft selbstständig
abgeschlossen werden. Eine Anmeldung in einer Kammer ist rechtlich nicht in jedem Fall erforderlich, jedoch sind Gründer in vielen 
Fällen verpflichtet Mitglied einer Kammer zu werden. Eine Kammer bietet dem Gründer den Mehrwert der Unterstützung bei Beratungs- und Weiterbilungsthemen.

Als letzter Schritt in der Prozesskette einer Unternehmensgründung steht die Einrichtung eines Geschäftskontos. Dies ist im Allgemeinen
für jede Art von Unternehmung sinnvoll, allerdings lediglich bei Kapitalgesellschaften erforderlich.

Um diesen Prozess der Unternehmensgründung zu digitalisieren und zu automatisieren, sind zahlreiche Schnittstellen zu den Behörden erforderlich.
Diese Schnittstellen basieren auf verschiedenen Technologie und Standards und werden nachfolgend dargelegt.

Zur zentralen Abwicklung von Steuerangelegenheiten dient die vom Finanzamt implementierte Schnittstelle "ELSTER".
Diese kann sowohl für die steuerliche Erfassung von Unternehmen, als auch für die Einreichung einer Steuererklärung dienen.
% mehr zu Elster erläutern

Obschon übergreifend und Deutschlandweit keine einheitliche Schnittstelle zur Gewerbeanmeldung existiert, 
bieten einige Bundesländer Portale zur digitalen Anmeldung von Gewerben an. 
% mehr zu den Portalen erläutern

Eine Handelsregisteranmeldung erfolgt weiterhin über einen Notar. Dieser leitet die Daten elektronisch an das Handelsregister weiter.
% mehr zur Handelsregisteranmeldung erläutern


