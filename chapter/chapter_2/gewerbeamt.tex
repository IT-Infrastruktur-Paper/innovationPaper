
Bei der Gründung eines Unternehmens sind wie bereits beschrieben mehrere Behörden
involviert, die unterschiedliche Rollen im Gründungsprozess erfüllen.
Um diesen Prozess der Unternehmensgründung zu digitalisieren und zu automatisieren, sind zahlreiche Schnittstellen zu den Behörden erforderlich.
Diese Schnittstellen basieren auf verschiedenen Technologie und Standards und werden nachfolgend dargelegt.

Der erste Schritt der Unternehmensgründung ist die Anmeldung eines Gewerbes. Dieser erfolgt
durch das Gewerbeamt. Zur Anmeldung wird dem Gründer ein Gewerbeschein angemeldet, der die Grundlage
für die Ausübung einer unternehmerischen Tätigkeit darstellt.

Obschon übergreifend und Deutschlandweit keine einheitliche Schnittstelle zur Gewerbeanmeldung existiert, 
bieten einige Bundesländer Portale zur digitalen Anmeldung von Gewerben an. 

Als letzter Schritt in der Prozesskette einer Unternehmensgründung steht die Einrichtung eines Geschäftskontos. Dies ist im Allgemeinen
für jede Art von Unternehmung sinnvoll, allerdings lediglich bei Kapitalgesellschaften erforderlich.


% mehr zur Handelsregisteranmeldung erläutern




