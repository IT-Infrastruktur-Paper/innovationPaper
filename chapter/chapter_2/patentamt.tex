
Neben der Anmeldung eines Gewerbes ist auch die elektronische Anmeldung eines Patents, Gebrauchsmusters oder die Eintragung einer 
Marke von Relevanz für die Umsetzung des vorgeschlagenen Produkts. Für die Anmeldung muss vorab allerdings der geografische Wirkungsbereich
der einzutragenden Schutzes geklärt werden. Dessen Gültigkeitsbereich kann sich entweder auf Deutschland oder die EU beschränken, oder 
weltweit geschützt sein.

Zur elektronischen Anmeldung von Marken und Designs sowie der internationalen Registrierung deutscher Marken ist das Deutsche Patent und Markenamt
(DPMA) zuständig. Auf der Internetseite der DPMA können über ein Onlineformular Marken und Designs sowie internationale Marken angemeldet werden.
Zur Anmeldung eines Patents durch das DPMA ist jedoch eine kostenlose Software erforderlich, wodurch sich diese nicht barrierefrei durchführen lässt.

Die Anmeldung von europäischen Patenten (EP) und Patent Cooperation Treaties (PCT) erfolgt in der EU durch das Europäische Patentamt.
Dieses stellt ein Online-Tool zur Anmeldung von EP und PCT bereit.
