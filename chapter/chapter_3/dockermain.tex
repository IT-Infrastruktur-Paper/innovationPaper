Für Docker und zur Erstellung von Containern ist jeweils im Frontend, Backend und in Sabbrands eine Dockerfile geschrieben, über die Docker Images erstellt werden. Sowohl Strapi im Backend, als auch alle Seiten im Frontend laufen jeweils in einem eigens isolierten Container. Die Container exposen dabei einen Port nach außen, um im Web erreichbar zu sein.

\begin{lstlisting}[language=JavaScript, caption={Dockerfile Strapi}, label={lst:dockerfileStrapi}]
FROM node:20              
WORKDIR /usr/src/app      
COPY package*.json ./     
RUN npm install           
COPY . .                  
ENV NODE_ENV=production
ENV ENV_PATH=./.env_prod
RUN npm run strapi build
RUN npm prune --production
EXPOSE 1337      
CMD ["npm", "run", "start"]
\end{lstlisting}

Hier sieht man eine beispielhafte Dockerfile für Strapi.