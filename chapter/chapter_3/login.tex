Eine Seite, bzw. Komponente ist die Login Seite, welche unter „rfinnotrade/frontend/src/components/pages/Login.jsx“ abgelegt ist.
Die Login.jsx Datei definiert die Login Seite unseres Frontends. Sie hat grundsätzlich die Funktion, den Benutzern die Anmeldung per E-Mail oder Benutzernamen mit dem dazugehörigen Passwort zu ermöglichen.

\begin{lstlisting}[language=JavaScript, caption={Login.jsx Frontendaufbau}, label={lst:loginjsxUI}]
return (
    <div className="login-container">
            <div className="login-header">
            <h1 className="login-title">RF InnoTrade Admin Center</h1>
        </div>
        <div className="login-background">
            <div className="login-box">
                <h2 className="welcome-back">Welcome Back</h2>
                <p className="instructions">Gib deine Zugangsdaten ein, um Zugang zu deinem Account zu erhalten</p>
                <form onSubmit={handleSubmit} className="login-form">
                    <div className="form-group">
                        <input
                            type="text"
                            placeholder="Gib deine Email oder Username ein"
                            value={identifier}
                            onChange={(e) => setIdentifier(e.target.value)}
                            required
                            className="form-input"
                        />
                    </div>
                    <div className="form-group">
                        <input
                            type="password"
                            placeholder="Gib dein Passwort ein"
                            value={password}
                            onChange={(e) => setPassword(e.target.value)}
                            required
                            className="form-input"
                        />
                    </div>
                    <button type="submit" className="login-button">Anmelden</button>
                </form>
                <p className="reset-password">
                    Passwort vergessen? 
                    {/* <Link to="/password-reset">Passwort zurücksetzen </Link> */}
                </p>
                {error && <p className="login-error">{error}</p>}
            </div>
        </div>
    </div>
);
\end{lstlisting}

Das User-Interface (UI) ist ebenfalls hier implentiert und besteht aus einem Eingabefeld für die E-Mail und den Benutzernamen, einem Eingabefeld für das Passwort und einer Schaltfläche zum Absenden der Login-Daten. Wenn der Login fehlschlät, erscheint eine Fehlermeldung unterhalb des Formulars.
Das Styling der Seite wird über eine ausgelagerte Login.css Datei gesteuert.
Um den Login eines Benutzers durchzuführen, werden in der Login Komponente verschiedene Informationen verwaltet

\begin{lstlisting}[language=JavaScript, caption={Login.jsx states}, label={lst:loginjsxStates}]
const [password, setPassword] = useState('');
const [identifier, setIdentifier] = useState('');
const [error, setError] = useState('');
\end{lstlisting}

Wie hier zu sehen ist, verwaltet die Login Komponente password, identifiert und error, um die Logindaten oder Fehler bei der Anmeldung innerhalb der Komponente zu speichern und verarbeiten zu können.
Sendet der Benutzer den Login per Schaltfläche ab, wird die Funktion „handleSubmit“ aufgerufen. Diese sendet einen POST-Request an die Login-API von Strapi, überprüft dort die Anmeldedaten und der Benutzer wird bei erfolg an die Admin-Seite der Webseite weitergeleitet.

\begin{lstlisting}[language=JavaScript, caption={Login.jsx Login-Funktion}, label={lst:loginjsxLoginFunktion}]
try {
    const response = await axios.post(import.meta.env.VITE_STRAPI_BASE_URL+'api/auth/login', {
        identifier,
        password,
    }, {
        withCredentials: true // This is important for including cookies in the request
    });
    console.log('User profile', response.data.user);
    navigate('/Admin-Center'); // Redirect to the home page
    props.onLoginSuccess();
    } catch (error) {
    setError('Login fehlgeschlagen bitte überprüfe deine Eingabe.');
    }
};
\end{lstlisting}

Außerdem wird die Funktion „onLoginSuccess“ ausgeführt, damit der Authentifizierungsstatus des Benutzers aktualisiert wird.